\documentclass[]{article}
\usepackage[a4paper, total={15cm,23cm}]{geometry}
\usepackage{fancyhdr}
\usepackage{graphicx}
\usepackage{amsmath}
\usepackage{amssymb}
%\usepackage[dvipsnames]{xcolor}
\usepackage{tikz}
\usepackage{verbatim}
\usepackage{tcolorbox}
\usepackage{textcomp}
\usepackage{xcomment}
\usepackage{xstring}
%opening
\title{LC Time Interpretation}
\author{Benjamin Bauml}
\date{Winter 2024}
\pagestyle{fancy}
\rhead{PH 223}
\chead{Winter 2024}
\lhead{Week 10}

% Version 2024-02-21
% Changes
% 2024-02-21 Added xstring package to enable smooth implementation of new \ModePage command.
% For Assignment, leave Purpose as 1. For Worksheet, set to 2. For Student Solution, set to 3. For Teacher Solution, set to 4.
\newcommand{\Purpose}{4}

\newcommand{\Exclusion}{0}
\newcommand{\PageTurn}{0}
\newcommand{\GrayProb}{0}
\newcommand{\Tipsy}{0}

% Assignment
\if\Purpose1
\renewcommand{\Exclusion}{1}
\fi
% Worksheet
\if\Purpose2
\renewcommand{\Exclusion}{1}
\renewcommand{\PageTurn}{1}
\fi
% Student Solution
\if\Purpose3
\renewcommand{\PageTurn}{1}
\renewcommand{\GrayProb}{1}
\fi
% Teaching Copy
\if\Purpose4
\renewcommand{\PageTurn}{1}
\renewcommand{\GrayProb}{1}
\renewcommand{\Tipsy}{1}
\fi

\if\Exclusion1
\xcomment{Title,Problem,ProblemSub,PassFig}
\fi

\def \NewQ {0}
\def \PForce {0}
\newcommand{\MaybePage}[1]{
	\def \PForce {#1}
	\if\PForce1
		\newpage
	\else
		\if\NewQ0
		\gdef \NewQ {\PageTurn}
		\else
		\newpage
		\fi
	\fi
}

\newcommand{\ModePage}[1]{
	\IfSubStr{#1}{\Purpose}{\newpage}{}
}

\newenvironment{Problem}[2][0]{%The first argument is optional, and if it is set to 1, the \newpage will be forced.
\MaybePage{#1}
\section*{#2}
\if\GrayProb1
\begin{tcolorbox}[colback=lightgray,colframe=lightgray,sharp corners,boxsep=1pt,left=0pt,right=0pt,top=0pt,bottom=0pt,after skip=2pt]
\else
\begin{tcolorbox}[colback=white,colframe=white,sharp corners,boxsep=1pt,left=0pt,right=0pt,top=0pt,bottom=0pt,after skip=2pt]
\fi
}{
\end{tcolorbox}\noindent
}

\newenvironment{ProblemSub}[1][0]{%The argument is optional, and if a string of numbers is entered into it, it will force a \newpage in any \Purpose that shows up in the string. For example, "13" would lead to the newpage being forced in modes 1 and 3.
\ModePage{#1}
\if\GrayProb1
\begin{tcolorbox}[colback=lightgray,colframe=lightgray,sharp corners,boxsep=1pt,left=0pt,right=0pt,top=0pt,bottom=0pt,after skip=2pt]
\else
\begin{tcolorbox}[colback=white,colframe=white,sharp corners,boxsep=1pt,left=0pt,right=0pt,top=0pt,bottom=0pt,after skip=2pt]
\fi
}{
\end{tcolorbox}\noindent
}

\newenvironment{PassFig}{\begin{figure}[h]}{\end{figure}}

\newcommand{\TeachingTips}[1]{
\if\Tipsy1
\begin{tcolorbox}[colback=lightgray,colframe=black]
#1
\end{tcolorbox}
\fi
}

\newenvironment{Title}{\maketitle}{}

\begin{document}
\begin{Title}
\begin{center}
	This problem is borrowed from Chapter 30 of the \textit{Student Workbook} for \textit{Physics for Scientists and Engineers}.
\end{center}
\end{Title}

\begin{Problem}{Activity}
	The capacitor in an $ LC $ circuit has maximum charge at $ t = 1 $ \textmu s. The current through the inductor next reaches a maximum at $ t = 3 $ \textmu s.
\end{Problem}
\begin{ProblemSub}
	(a) When will the inductor current reach a maximum in the opposite direction?
\end{ProblemSub}
The current in an $ LC $ circuit oscillates as $ I = I_{max}\sin(\omega t + \phi) $, where $ \phi $ is some phase constant. The charge on the capacitor is maximum when the current is zero, so the two events are 90$ ^{\circ} $ out of phase. That means the time between the two is a quarter of the circuit's period:
\[
\Delta t = 2\text{ \textmu s} = \frac{T}{4}.
\]
The inductor current changes direction after a 180$ ^{\circ} $ phase change, so we need to add half of the period ($ T/2 = 4 $ \textmu s) to the time of the first maximum ($ t = 3 $ \textmu s) to reach the inverted maximum. That gives us the desired time of $ t = 7 $ \textmu s.
\begin{ProblemSub}
	(b) What is the circuit's period of oscillation?
\end{ProblemSub}
As we said before, $ \Delta t = \frac{T}{4} $, so $ T = 8 $ \textmu s.
\end{document}
