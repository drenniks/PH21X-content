\documentclass[]{article}
\newcommand{\FileDepth}{../../..}
\usepackage[letterpaper, landscape, margin=0.5cm]{geometry}
\usepackage[T1]{fontenc}
\usepackage{textcomp}%Not strictly necessary, but gives \textmu command for "micro."
\usepackage{fancyhdr}
\usepackage{amsmath}
\usepackage{amssymb}
\usepackage{graphicx}
\usepackage{xcolor}
\usepackage{tikz}
\usetikzlibrary{calc}
\usepackage[shortlabels]{enumitem}
\usepackage{multicol}
\usepackage{vwcol}
\usepackage{hyperref}
\usepackage{wrapfig}
%opening
\newcommand{\SecType}{L}
\newcommand{\Week}{3}
\title{PH 21X Lecture \Week}
\author{Benjamin Bauml}
\date{Term 2024}

\newcommand{\Purpose}{4}
\newcommand{\DefOnly}{0}

\input{\FileDepth/Formats/Assignment20240614.tex}
\usepackage[absolute]{textpos}
% This package relies on Assignment Format 2024-06-14 or later to work. It is recommended that the Purpose and DefOnly commands be given as such:
%\newcommand{\Purpose}{4}
%\newcommand{\DefOnly}{0}
% Activities need to be entered outside of the TeacherMargin and PresentSpace environments, otherwise they will be defined only locally. They can even go in the preamble.
\newenvironment{TeacherMargin}{\begin{textblock*}{10.8cm}(0.5cm,0.5cm)
\small}{\end{textblock*}
\hspace{0.1cm}}
\newenvironment{PresentSpace}{\begin{textblock*}{0.3cm}(26.85cm,9.35cm)
--
\end{textblock*}
\begin{textblock*}{15.6cm}(11.8cm,0.5cm)
\begin{Repurpose}{1}
\Large}{\end{Repurpose}
\end{textblock*}
\hspace{0.1cm}}

%\newcommand{\FBDaxes}[4][2]{
	\begin{scope}[shift={(#2)},rotate=#3]
		% x-axis
		\draw[thick,->] (-#1,0) -- (#1,0);
		\node[anchor=west] at (#1,0) {$x$};
		% y-axis
		\draw[thick,->] (0,-#1) -- (0,#1);
		\node[anchor=south] at (0,#1) {$y$};
		\coordinate (#4) at (0,0);
	\end{scope}
}
\newcommand{\FBDvectorMA}[4]{
	\begin{scope}[shift={(#1)}]
		\coordinate (#4tip) at ({#2*cos(#3)},{#2*sin(#3)});
		\draw[ultra thick,blue,->] (#1) -- (#4tip);
	\end{scope}
}
\newcommand{\FBDvectorXY}[3]{
	\begin{scope}[shift={(#1)}]
		\coordinate (#3tip) at (#2);
		\draw[ultra thick,blue,->] (0,0) -- (#3tip);
	\end{scope}
}
\newcommand{\FBDdot}[1]{
	\filldraw[black] (#1) circle (3pt);
}
\newcommand{\FBDbox}[5][1]{
	\begin{scope}[shift={(#2)},rotate=#3]
		\filldraw[color=black,fill=white,thick] ({-#1/2},{#1/2}) -- ({-#1/2},{-#1/2}) -- ({#1/2},{-#1/2}) -- ({#1/2},{#1/2}) -- cycle;
		% Left side coordinates
		\coordinate (#4ltq) at ({-#1/2},{#1/4});
		\coordinate (#4lcent) at ({-#1/2},0);
		\coordinate (#4lbq) at ({-#1/2},{-#1/4});
		% right side coordinates
		\coordinate (#4rtq) at ({#1/2},{#1/4});
		\coordinate (#4rcent) at ({#1/2},0);
		\coordinate (#4rbq) at ({#1/2},{-#1/4});
		% top coordinates
		\coordinate (#4tlq) at ({-#1/4},{#1/2});
		\coordinate (#4tcent) at (0,{#1/2});
		\coordinate (#4trq) at ({#1/4},{#1/2});
		% bottom coordinates
		\coordinate (#4blq) at ({-#1/4},{-#1/2});
		\coordinate (#4bcent) at (0,{-#1/2});
		\coordinate (#4brq) at ({#1/4},{-#1/2});
		% corners
		\coordinate (#4tl) at ({-#1/2},{#1/2});
		\coordinate (#4tr) at ({#1/2},{#1/2});
		\coordinate (#4bl) at ({-#1/2},{-#1/2});
		\coordinate (#4br) at ({#1/2},{-#1/2});
		\node at (0,0) {#5};
	\end{scope}
}
%\newcommand{\MVec}[3][0]{%Creates a momentum vector of length #3 centered at #2 and rotated #1 degrees counterclockwise.
	\begin{scope}[rotate=#1,shift={(#2)}]
		\draw[->,thick] ({-#3/2},0) -- ({#3/2},0);
	\end{scope}
}
\newcommand{\MDot}[1]{%Creates a dot at #1 to represent a zero vector.
	\filldraw (#1) circle (1pt);
}
\newcommand{\MVDRows}[2][4.5]{%Creates the rows (initial, delta, final) of a momentum vector diagram. The optional argument determines the width of the table, and defaults to a good length for three columns (two objects and the total system). The non-optional argument gives a coordinate name (not displayed) to the diagram.
	\begin{scope}
		%\draw[thick] (0,5.5) -- (0,0);
		\draw[thick] (-1,4.5) -- (#1,4.5);
		\node at (-0.5,3.75) {$\vec{p}_{i}$};
		\draw[thick] (-1,3) -- (#1,3);
		\node at (-0.5,2.25) {$\Delta\vec{p}$};
		\draw[thick] (-1,1.5) -- (#1,1.5);
		\node at (-0.5,0.75) {$\vec{p}_{f}$};
		\coordinate (#2) at (0,5);
	\end{scope}
}
\newcommand{\MVDCol}[4][0.75]{%Creates a column for an object in a momentum vector diagram. The first (non-optional) argument is the coordinate name (not displayed) of the column, while the second is the displayed column header. The first argument also names the three entries down the column. The third argument anchors the column, so it should either be the coordinate name of the MVD (for the first column) or the coordinate name of the previous column. The optional argument indicates how far the center of the column should be from the previous column's edge, and defaults to 0.75.
	\begin{scope}[shift={(#4)}]
		\node at (#1,0) {#3};
		%\draw[thick] ({#1*2},0.5) -- ({#1*2},-5);
		\draw[thick] (0,0.5) -- (0,-5);
		\coordinate (#2init) at (#1,-1.25);
		\coordinate (#2delt) at (#1,-2.75);
		\coordinate (#2fin) at (#1,-4.25);
		\coordinate (#2) at ({#1*2},0);
	\end{scope}
}

%\input{\FileDepth/Activities/Activity_One/Activity_One.tex}
%\input{\FileDepth/Activities/Activity_Two/Activity_Two.tex}

\begin{document}
\begin{TeacherMargin}

\end{TeacherMargin}
\begin{PresentSpace}
\begin{center}
	\huge Lecture 4: Using Integrals in Physics
\end{center}
\vspace{1cm}
\underline{Warm-Up Activity} \\
How is acceleration symbolically related to velocity?
\begin{enumerate}[(A)]
	\item Velocity is acceleration times $t$.
	\item Acceleration is velocity times $t$.
	\item Acceleration is the derivative of velocity.
	\item Velocity is the derivative of acceleration.
\end{enumerate}
\end{PresentSpace}
\newpage
\begin{TeacherMargin}

\end{TeacherMargin}
\begin{PresentSpace}
\vspace{-10pt}
\section*{L4-1: Vax'ildan's Acceleration}
\vspace{-20pt}
\begin{multicols}{2}
\begin{itemize}
	\item Vax'ildan Vessar is initially located at position $x_{i}$, running to the right with initial speed $v_{i}$.
	\item At $t=0$, Vax clicks his \textit{boots of haste}, which provide an acceleration:
	\vspace{-10pt}
	\[
	\vec{a}(t) = a_{0}\left(1-\frac{t}{T}\right)\hat{x}
	\]
	\vspace{-20pt}
	\item Our goals are:
	\begin{itemize}
		\item Find how much time it takes for Vax to return to his initial velocity.
		\item Find Vax's position at this time.
	\end{itemize}
\end{itemize}
\begin{center}
\reflectbox{\includegraphics[scale=0.15]{Vax-Linda_Lithen}}
\end{center}
\end{multicols}
%\vspace{1.5cm}
\section*{Solving an ARCS Problem}
\begin{center}
	\includegraphics[scale=0.4]{AnalyzeAndRepresent}
	\includegraphics[scale=0.4]{Calculate}
	\includegraphics[scale=0.4]{Sensemake}
\end{center}
\end{PresentSpace}
\newpage
\begin{TeacherMargin}
We need $1-\frac{t}{T}$ to be unitless, because 1 is unitless, so $t/T$ must also be. Since $t$ is in seconds, it follows that $T$ must also be.

Note that, when we plug in $t = T$, we find that $\vec{a}(T) = 0\hat{x}$, so the acceleration burst stops after $T$ passes, at which point the acceleration changes direction to bring Vax back to his initial velocity. As such, $T$ can be thought of as the duration of the acceleration burst.

Since $\left(1-\frac{t}{T}\right)\hat{x}$ is unitless, the overall units of the right hand side come from $a_{0}$. We need the right hand side to be an acceleration to match the left, so $a_{0}$ has units of m/s$^{2}$.

Note that, when we plug in $t=0$ s, we find that $\vec{a}(0\text{ s})=a_{0}\hat{x}$, so $a_{0}$ is the magnitude of the initial acceleration.

The unit vector $\hat{x}$ carries all of the direction information. It tells us that the acceleration is in the $x$-direction (though left or right depends on the sign).

\noindent\textbf{Understand and Plan}
\vspace{-10pt}
\begin{multicols}{2}
	\noindent\textbf{Knowns}
	\begin{itemize}
		\item Initial Position: $x_{i} = 0$ m (simplifying assumption)
		\item Initial Velocity: $v_{i} = 2$ m/s (a reasonable speed to estimate for a half-elf)
		\item $T = 6$ s (rounds in \textit{Dungeons \& Dragons} last six seconds)
		\item $a_{0} = 0.5$ m/s$^{2}$ (significantly less than free-fall acceleration)
	\end{itemize}
	\noindent\textbf{Unknowns}
	\begin{itemize}
		\item When Vax returns to his initial velocity: $t_{f}$
		\item Vax's final position: $x_{f}$
		\item Equations of motion for velocity and position: $\vec{v}(t)$ and $\vec{x}(t)$
	\end{itemize}
\end{multicols}
\vspace{-10pt}
\noindent\textbf{Identify Assumptions}
\begin{itemize}
	\item Particle Model
	\begin{itemize}
		\item We do not wish to handle the complexities of how Vax's arms, legs, and wings move as he runs, so we will treat him as a point mass.
	\end{itemize}
	\item 1-D Motion
	\begin{itemize}
		\item We will assume Vax is travelling over relatively level ground so we do not have to consider Vax's vertical motion.
	\end{itemize}
	\item Vax is not obstructed in his movement.
	\begin{itemize}
		\item If Vax were to bump into something, that brief interaction would alter his motion in additional ways that the acceleration of the boots does not account for.
	\end{itemize}
\end{itemize}
\noindent\textbf{Represent Physically}
\vspace{-10pt}
\begin{center}
	\begin{tikzpicture}
		\draw[thick,->] (0,-2.2) -- (0,2.2) node[anchor=south] {$a(t)$};
		\draw[thick,->] (0,0) -- (4.2,0) node[anchor=west] {$t$};
		\draw[thick] (2,4pt) -- (2,-4pt) node[anchor=north] {$T$};
		\draw[thick] (4,4pt) -- (4,-4pt) node[anchor=north] {$2T$};
		\draw[thick] (-4pt,2) node[anchor=east] {$a_{0}$} -- (0,2);
		\draw[thick,blue] (0,2) -- (4,-2);
	\end{tikzpicture}
\end{center}
\end{TeacherMargin}
\begin{PresentSpace}
\vspace{-10pt}
\section*{L4-1: Vax'ildan's Acceleration -- Calculate}
\vspace{-10pt}
%\begin{multicols}{2}
\begin{itemize}
	\item At $t=0$, Vax clicks his \textit{boots of haste}, which provide an acceleration:
	\[
	\vec{a}(t) = a_{0}\left(1-\frac{t}{T}\right)\hat{x}
	\]
	%\vspace{40pt}
	%\item With your group:
	\begin{itemize}
		\item First find a symbolic expression for Vax's velocity as a function of time.
		\item Use your expression to find when Vax's velocity is equal to $v_{i}$.
		\item Estimate any quantities to find numerical answers.
	\end{itemize}
\end{itemize}
%\end{multicols}
\end{PresentSpace}
\newpage
\begin{TeacherMargin}
We need $1-\frac{t}{T}$ to be unitless, because 1 is unitless, so $t/T$ must also be. Since $t$ is in seconds, it follows that $T$ must also be.

Note that, when we plug in $t = T$, we find that $\vec{a}(T) = 0\hat{x}$, so the acceleration burst stops after $T$ passes, at which point the acceleration changes direction to bring Vax back to his initial velocity. As such, $T$ can be thought of as the duration of the acceleration burst.

Since $\left(1-\frac{t}{T}\right)\hat{x}$ is unitless, the overall units of the right hand side come from $a_{0}$. We need the right hand side to be an acceleration to match the left, so $a_{0}$ has units of m/s$^{2}$.

Note that, when we plug in $t=0$ s, we find that $\vec{a}(0\text{ s})=a_{0}\hat{x}$, so $a_{0}$ is the magnitude of the initial acceleration.

The unit vector $\hat{x}$ carries all of the direction information. It tells us that the acceleration is in the $x$-direction (though left or right depends on the sign).

\noindent\textbf{Understand and Plan}
\vspace{-10pt}
\begin{multicols}{2}
	\noindent\textbf{Knowns}
	\begin{itemize}
		\item Initial Position: $x_{i} = 0$ m (simplifying assumption)
		\item Initial Velocity: $v_{i} = 2$ m/s (a reasonable speed to estimate for a half-elf)
		\item $T = 6$ s (rounds in \textit{Dungeons \& Dragons} last six seconds)
		\item $a_{0} = 0.5$ m/s$^{2}$ (significantly less than free-fall acceleration)
	\end{itemize}
	\noindent\textbf{Unknowns}
	\begin{itemize}
		\item When Vax returns to his initial velocity: $t_{f}$
		\item Vax's final position: $x_{f}$
		\item Equations of motion for velocity and position: $\vec{v}(t)$ and $\vec{x}(t)$
	\end{itemize}
\end{multicols}
\vspace{-10pt}
\noindent\textbf{Identify Assumptions}
\begin{itemize}
	\item Particle Model
	\begin{itemize}
		\item We do not wish to handle the complexities of how Vax's arms, legs, and wings move as he runs, so we will treat him as a point mass.
	\end{itemize}
	\item 1-D Motion
	\begin{itemize}
		\item We will assume Vax is travelling over relatively level ground so we do not have to consider Vax's vertical motion.
	\end{itemize}
	\item Vax is not obstructed in his movement.
	\begin{itemize}
		\item If Vax were to bump into something, that brief interaction would alter his motion in additional ways that the acceleration of the boots does not account for.
	\end{itemize}
\end{itemize}
\noindent\textbf{Represent Physically}
\vspace{-10pt}
\begin{center}
	\begin{tikzpicture}
		\draw[thick,->] (0,-2.2) -- (0,2.2) node[anchor=south] {$a(t)$};
		\draw[thick,->] (0,0) -- (4.2,0) node[anchor=west] {$t$};
		\draw[thick] (2,4pt) -- (2,-4pt) node[anchor=north] {$T$};
		\draw[thick] (4,4pt) -- (4,-4pt) node[anchor=north] {$2T$};
		\draw[thick] (-4pt,2) node[anchor=east] {$a_{0}$} -- (0,2);
		\draw[thick,blue] (0,2) -- (4,-2);
	\end{tikzpicture}
\end{center}
\end{TeacherMargin}
\begin{PresentSpace}
\vspace{-10pt}
\section*{L4-1: Vax'ildan's Acceleration -- Calculate}
\vspace{-10pt}
%\begin{multicols}{2}
\begin{itemize}
	\item At $t=0$, Vax clicks his \textit{boots of haste}, which provide an acceleration:
	\[
	\vec{a}(t) = a_{0}\left(1-\frac{t}{T}\right)\hat{x}
	\]
	\item His velocity as a function of time is
	\[
	\vec{v}(t) = \left[v_{i}+a_{0}\left(t-\frac{t^{2}}{2T}\right)\right]\hat{x},
	\]
	and he returns to his initial velocity at $t_{f}=2T$.
	%\vspace{40pt}
	%\item With your group:
	\begin{itemize}
		\item Now find a symbolic expression for Vax's position as a function of time and use it to find Vax's position at $t_{f}$.
	\end{itemize}
\end{itemize}
%\end{multicols}
\end{PresentSpace}
\newpage
\begin{TeacherMargin}
We need $1-\frac{t}{T}$ to be unitless, because 1 is unitless, so $t/T$ must also be. Since $t$ is in seconds, it follows that $T$ must also be.

Note that, when we plug in $t = T$, we find that $\vec{a}(T) = 0\hat{x}$, so the acceleration burst stops after $T$ passes, at which point the acceleration changes direction to bring Vax back to his initial velocity. As such, $T$ can be thought of as the duration of the acceleration burst.

Since $\left(1-\frac{t}{T}\right)\hat{x}$ is unitless, the overall units of the right hand side come from $a_{0}$. We need the right hand side to be an acceleration to match the left, so $a_{0}$ has units of m/s$^{2}$.

Note that, when we plug in $t=0$ s, we find that $\vec{a}(0\text{ s})=a_{0}\hat{x}$, so $a_{0}$ is the magnitude of the initial acceleration.

The unit vector $\hat{x}$ carries all of the direction information. It tells us that the acceleration is in the $x$-direction (though left or right depends on the sign).

\noindent\textbf{Understand and Plan}
\vspace{-10pt}
\begin{multicols}{2}
	\noindent\textbf{Knowns}
	\begin{itemize}
		\item Initial Position: $x_{i} = 0$ m (simplifying assumption)
		\item Initial Velocity: $v_{i} = 2$ m/s (a reasonable speed to estimate for a half-elf)
		\item $T = 6$ s (rounds in \textit{Dungeons \& Dragons} last six seconds)
		\item $a_{0} = 0.5$ m/s$^{2}$ (significantly less than free-fall acceleration)
	\end{itemize}
	\noindent\textbf{Unknowns}
	\begin{itemize}
		\item When Vax returns to his initial velocity: $t_{f}$
		\item Vax's final position: $x_{f}$
		\item Equations of motion for velocity and position: $\vec{v}(t)$ and $\vec{x}(t)$
	\end{itemize}
\end{multicols}
\vspace{-10pt}
\noindent\textbf{Identify Assumptions}
\begin{itemize}
	\item Particle Model
	\begin{itemize}
		\item We do not wish to handle the complexities of how Vax's arms, legs, and wings move as he runs, so we will treat him as a point mass.
	\end{itemize}
	\item 1-D Motion
	\begin{itemize}
		\item We will assume Vax is travelling over relatively level ground so we do not have to consider Vax's vertical motion.
	\end{itemize}
	\item Vax is not obstructed in his movement.
	\begin{itemize}
		\item If Vax were to bump into something, that brief interaction would alter his motion in additional ways that the acceleration of the boots does not account for.
	\end{itemize}
\end{itemize}
\noindent\textbf{Represent Physically}
\vspace{-10pt}
\begin{center}
	\begin{tikzpicture}
		\draw[thick,->] (0,-2.2) -- (0,2.2) node[anchor=south] {$a(t)$};
		\draw[thick,->] (0,0) -- (4.2,0) node[anchor=west] {$t$};
		\draw[thick] (2,4pt) -- (2,-4pt) node[anchor=north] {$T$};
		\draw[thick] (4,4pt) -- (4,-4pt) node[anchor=north] {$2T$};
		\draw[thick] (-4pt,2) node[anchor=east] {$a_{0}$} -- (0,2);
		\draw[thick,blue] (0,2) -- (4,-2);
	\end{tikzpicture}
\end{center}
\end{TeacherMargin}
\begin{PresentSpace}
\vspace{-10pt}
\section*{L4-1: Vax'ildan's Acceleration -- Sensemake}
\vspace{-10pt}
%\begin{multicols}{2}
\begin{itemize}
	\item At $t=0$, Vax clicks his \textit{boots of haste}, which provide an acceleration:
	\[
	\vec{a}(t) = a_{0}\left(1-\frac{t}{T}\right)\hat{x}
	\]
	%\vspace{40pt}
	%\item With your group:
	\begin{itemize}
		\item How can we make sense of these equations?
		\begin{align*}
		\vec{v}(t) & = \left[v_{i}+a_{0}\left(t-\frac{t^{2}}{2T}\right)\right]\hat{x} & \vec{x}(t) & = \left[x_{i}+v_{i}t+a_{0}\left(\frac{t^{2}}{2}-\frac{t^{3}}{6T}\right)\right]\hat{x} \\
		t_{f} & = 2T & \vec{x}_{f} & = \left[x_{i}+2v_{i}T+\frac{2}{3}a_{0}T^{2}\right]\hat{x}
		\end{align*}
	\end{itemize}
\end{itemize}
%\end{multicols}
\end{PresentSpace}
\newpage
\begin{TeacherMargin}
We need $1-\frac{t}{T}$ to be unitless, because 1 is unitless, so $t/T$ must also be. Since $t$ is in seconds, it follows that $T$ must also be.

Note that, when we plug in $t = T$, we find that $\vec{a}(T) = 0\hat{x}$, so the acceleration burst stops after $T$ passes, at which point the acceleration changes direction to bring Vax back to his initial velocity. As such, $T$ can be thought of as the duration of the acceleration burst.

Since $\left(1-\frac{t}{T}\right)\hat{x}$ is unitless, the overall units of the right hand side come from $a_{0}$. We need the right hand side to be an acceleration to match the left, so $a_{0}$ has units of m/s$^{2}$.

Note that, when we plug in $t=0$ s, we find that $\vec{a}(0\text{ s})=a_{0}\hat{x}$, so $a_{0}$ is the magnitude of the initial acceleration.

The unit vector $\hat{x}$ carries all of the direction information. It tells us that the acceleration is in the $x$-direction (though left or right depends on the sign).

\noindent\textbf{Understand and Plan}
\vspace{-10pt}
\begin{multicols}{2}
	\noindent\textbf{Knowns}
	\begin{itemize}
		\item Initial Position: $x_{i} = 0$ m (simplifying assumption)
		\item Initial Velocity: $v_{i} = 2$ m/s (a reasonable speed to estimate for a half-elf)
		\item $T = 6$ s (rounds in \textit{Dungeons \& Dragons} last six seconds)
		\item $a_{0} = 0.5$ m/s$^{2}$ (significantly less than free-fall acceleration)
	\end{itemize}
	\noindent\textbf{Unknowns}
	\begin{itemize}
		\item When Vax returns to his initial velocity: $t_{f}$
		\item Vax's final position: $x_{f}$
		\item Equations of motion for velocity and position: $\vec{v}(t)$ and $\vec{x}(t)$
	\end{itemize}
\end{multicols}
\vspace{-10pt}
\noindent\textbf{Identify Assumptions}
\begin{itemize}
	\item Particle Model
	\begin{itemize}
		\item We do not wish to handle the complexities of how Vax's arms, legs, and wings move as he runs, so we will treat him as a point mass.
	\end{itemize}
	\item 1-D Motion
	\begin{itemize}
		\item We will assume Vax is travelling over relatively level ground so we do not have to consider Vax's vertical motion.
	\end{itemize}
	\item Vax is not obstructed in his movement.
	\begin{itemize}
		\item If Vax were to bump into something, that brief interaction would alter his motion in additional ways that the acceleration of the boots does not account for.
	\end{itemize}
\end{itemize}
\noindent\textbf{Represent Physically}
\vspace{-10pt}
\begin{center}
	\begin{tikzpicture}
		\draw[thick,->] (0,-2.2) -- (0,2.2) node[anchor=south] {$a(t)$};
		\draw[thick,->] (0,0) -- (4.2,0) node[anchor=west] {$t$};
		\draw[thick] (2,4pt) -- (2,-4pt) node[anchor=north] {$T$};
		\draw[thick] (4,4pt) -- (4,-4pt) node[anchor=north] {$2T$};
		\draw[thick] (-4pt,2) node[anchor=east] {$a_{0}$} -- (0,2);
		\draw[thick,blue] (0,2) -- (4,-2);
	\end{tikzpicture}
\end{center}
\end{TeacherMargin}
\begin{PresentSpace}
\vspace{-10pt}
\section*{L4-1: Vax'ildan's Acceleration -- Sensemake}
\vspace{-10pt}
\begin{itemize}
	\item How can we make sense of these equations?
	\begin{align*}
		\vec{v}(t) & = \left[v_{i}+a_{0}\left(t-\frac{t^{2}}{2T}\right)\right]\hat{x} & \vec{x}(t) & = \left[x_{i}+v_{i}t+a_{0}\left(\frac{t^{2}}{2}-\frac{t^{3}}{6T}\right)\right]\hat{x} \\
		t_{f} & = 2T & \vec{x}_{f} & = \left[x_{i}+2v_{i}T+\frac{2}{3}a_{0}T^{2}\right]\hat{x}
	\end{align*}
	\begin{itemize}
		\item Are the units correct?
		\item Which things are vectors?
		\item What do the graphs of $\vec{v}(t)$ and $\vec{x}(t)$ look like?
		\item What happens if you change $a_{0}$ or T?
		\item Try plugging in some reasonable numbers: $v_{i} = 2 \text{ m/s},\ T = 8 \text{ s},\ a_{0} = 0.5 \text{ m/s}^{2}, x_{i} = 15 \text{ m}$.
	\end{itemize}
\end{itemize}
\end{PresentSpace}
\newpage
\begin{TeacherMargin}
We need $1-\frac{t}{T}$ to be unitless, because 1 is unitless, so $t/T$ must also be. Since $t$ is in seconds, it follows that $T$ must also be.

Note that, when we plug in $t = T$, we find that $\vec{a}(T) = 0\hat{x}$, so the acceleration burst stops after $T$ passes, at which point the acceleration changes direction to bring Vax back to his initial velocity. As such, $T$ can be thought of as the duration of the acceleration burst.

Since $\left(1-\frac{t}{T}\right)\hat{x}$ is unitless, the overall units of the right hand side come from $a_{0}$. We need the right hand side to be an acceleration to match the left, so $a_{0}$ has units of m/s$^{2}$.

Note that, when we plug in $t=0$ s, we find that $\vec{a}(0\text{ s})=a_{0}\hat{x}$, so $a_{0}$ is the magnitude of the initial acceleration.

The unit vector $\hat{x}$ carries all of the direction information. It tells us that the acceleration is in the $x$-direction (though left or right depends on the sign).

\noindent\textbf{Understand and Plan}
\vspace{-10pt}
\begin{multicols}{2}
	\noindent\textbf{Knowns}
	\begin{itemize}
		\item Initial Position: $x_{i} = 0$ m (simplifying assumption)
		\item Initial Velocity: $v_{i} = 2$ m/s (a reasonable speed to estimate for a half-elf)
		\item $T = 6$ s (rounds in \textit{Dungeons \& Dragons} last six seconds)
		\item $a_{0} = 0.5$ m/s$^{2}$ (significantly less than free-fall acceleration)
	\end{itemize}
	\noindent\textbf{Unknowns}
	\begin{itemize}
		\item When Vax returns to his initial velocity: $t_{f}$
		\item Vax's final position: $x_{f}$
		\item Equations of motion for velocity and position: $\vec{v}(t)$ and $\vec{x}(t)$
	\end{itemize}
\end{multicols}
\vspace{-10pt}
\noindent\textbf{Identify Assumptions}
\begin{itemize}
	\item Particle Model
	\begin{itemize}
		\item We do not wish to handle the complexities of how Vax's arms, legs, and wings move as he runs, so we will treat him as a point mass.
	\end{itemize}
	\item 1-D Motion
	\begin{itemize}
		\item We will assume Vax is travelling over relatively level ground so we do not have to consider Vax's vertical motion.
	\end{itemize}
	\item Vax is not obstructed in his movement.
	\begin{itemize}
		\item If Vax were to bump into something, that brief interaction would alter his motion in additional ways that the acceleration of the boots does not account for.
	\end{itemize}
\end{itemize}
\noindent\textbf{Represent Physically}
\vspace{-10pt}
\begin{center}
	\begin{tikzpicture}
		\draw[thick,->] (0,-2.2) -- (0,2.2) node[anchor=south] {$a(t)$};
		\draw[thick,->] (0,0) -- (4.2,0) node[anchor=west] {$t$};
		\draw[thick] (2,4pt) -- (2,-4pt) node[anchor=north] {$T$};
		\draw[thick] (4,4pt) -- (4,-4pt) node[anchor=north] {$2T$};
		\draw[thick] (-4pt,2) node[anchor=east] {$a_{0}$} -- (0,2);
		\draw[thick,blue] (0,2) -- (4,-2);
	\end{tikzpicture}
\end{center}
\end{TeacherMargin}
\begin{PresentSpace}
\vspace{-10pt}
\section*{L4-2: Constant Acceleration}
\vspace{-10pt}
\begin{itemize}
	\item What if Vax's acceleration had been constant?
	\[
	\vec{a}(t) = a\hat{x}
	\]
\end{itemize}
\end{PresentSpace}
\newpage
\begin{TeacherMargin}

\end{TeacherMargin}
\begin{PresentSpace}
\section*{Main Ideas}
\begin{itemize}
	\item If we know the acceleration of an object as a function of time, we can determine the velocity as a function of time.
	\item If we know the velocity as a function of time, we can determine the position as a function of time.
\end{itemize}
\end{PresentSpace}
\end{document}