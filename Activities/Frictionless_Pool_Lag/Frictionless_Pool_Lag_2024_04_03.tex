\documentclass[]{article}
\usepackage[a4paper, total={15cm,23cm}]{geometry}
\usepackage{fancyhdr}
\usepackage{graphicx}
\usepackage{amsmath}
\usepackage{amssymb}
\usepackage{xcolor}
\usepackage{tikz}
\usepackage{verbatim}
\usepackage{tcolorbox}
\usepackage{textcomp}
\usepackage{xcomment}
\usepackage{xstring}
\usepackage{array}
%opening
\title{Frictionless Pool Lag}
\author{Benjamin Bauml}
\date{Spring 2024}
\pagestyle{fancy}
\rhead{PH 221}
\chead{Spring 2024}
\lhead{Week 1}

% Version 2024-02-21
% Changes
% 2024-02-21 Added xstring package to enable smooth implementation of new \ModePage command.
% For Assignment, leave Purpose as 1. For Worksheet, set to 2. For Student Solution, set to 3. For Teacher Solution, set to 4.
\newcommand{\Purpose}{4}

\newcommand{\Exclusion}{0}
\newcommand{\PageTurn}{0}
\newcommand{\GrayProb}{0}
\newcommand{\Tipsy}{0}

% Assignment
\if\Purpose1
\renewcommand{\Exclusion}{1}
\fi
% Worksheet
\if\Purpose2
\renewcommand{\Exclusion}{1}
\renewcommand{\PageTurn}{1}
\fi
% Student Solution
\if\Purpose3
\renewcommand{\PageTurn}{1}
\renewcommand{\GrayProb}{1}
\fi
% Teaching Copy
\if\Purpose4
\renewcommand{\PageTurn}{1}
\renewcommand{\GrayProb}{1}
\renewcommand{\Tipsy}{1}
\fi

\if\Exclusion1
\xcomment{Title,Problem,ProblemSub,PassFig}
\fi

\def \NewQ {0}
\def \PForce {0}
\newcommand{\MaybePage}[1]{
	\def \PForce {#1}
	\if\PForce1
		\newpage
	\else
		\if\NewQ0
		\gdef \NewQ {\PageTurn}
		\else
		\newpage
		\fi
	\fi
}

\newcommand{\ModePage}[1]{
	\IfSubStr{#1}{\Purpose}{\newpage}{}
}

\newenvironment{Problem}[2][0]{%The first argument is optional, and if it is set to 1, the \newpage will be forced.
\MaybePage{#1}
\section*{#2}
\if\GrayProb1
\begin{tcolorbox}[colback=lightgray,colframe=lightgray,sharp corners,boxsep=1pt,left=0pt,right=0pt,top=0pt,bottom=0pt,after skip=2pt]
\else
\begin{tcolorbox}[colback=white,colframe=white,sharp corners,boxsep=1pt,left=0pt,right=0pt,top=0pt,bottom=0pt,after skip=2pt]
\fi
}{
\end{tcolorbox}\noindent
}

\newenvironment{ProblemSub}[1][0]{%The argument is optional, and if a string of numbers is entered into it, it will force a \newpage in any \Purpose that shows up in the string. For example, "13" would lead to the newpage being forced in modes 1 and 3.
\ModePage{#1}
\if\GrayProb1
\begin{tcolorbox}[colback=lightgray,colframe=lightgray,sharp corners,boxsep=1pt,left=0pt,right=0pt,top=0pt,bottom=0pt,after skip=2pt]
\else
\begin{tcolorbox}[colback=white,colframe=white,sharp corners,boxsep=1pt,left=0pt,right=0pt,top=0pt,bottom=0pt,after skip=2pt]
\fi
}{
\end{tcolorbox}\noindent
}

\newenvironment{PassFig}{\begin{figure}[h]}{\end{figure}}

\newcommand{\TeachingTips}[1]{
\if\Tipsy1
\begin{tcolorbox}[colback=lightgray,colframe=black]
#1
\end{tcolorbox}
\fi
}

\newenvironment{Title}{\maketitle}{}

\begin{document}
\begin{Title}
\begin{center}
	This material is borrowed/adapted from PH 201 Tutorial 1 for Fall 2020.
\end{center}
\end{Title}

\begin{Problem}{Activity}
	You are about to play a game of pool, so you and your opponent are lagging for the first shot. To lag, you must bounce the cue ball off of the far end of the table, and get it as close to the near end as possible without touching it. While preparing to strike, you accidentally give the cue ball a light tap, setting it in motion.
	
	Uh-oh! Through some bizarre accident, the pool table has become entirely frictionless! The cue ball slides toward the opposite end without rolling or slowing down.
	
	A spectator across the table from you decides that the midpoint is $ x=0 $ cm. Relative to her, the edge near you is at $ x = 127 $ cm, and the far edge is at $ x = -127 $ cm.
	
	It takes the cue ball 4 s to slide at constant velocity from $ x = 89 $ cm to $ x = 17 $ cm.
\end{Problem}
\begin{ProblemSub}
	a) What is its velocity?
\end{ProblemSub}
Since the ball is traveling at constant speed, we can find its velocity exactly by dividing the ball's displacement (from 89 cm to 17 cm) by the time it takes to undergo that displacement:
\[
v = \frac{\Delta x}{\Delta t} = \frac{x_{f}-x_{i}}{\Delta t} = \frac{17\text{ cm} - 89\text{ cm}}{4\text{ s}} = \frac{-72\text{ cm}}{4\text{ s}} = -18\text{ cm/s.}
\]
Note the sign; velocity is a vector, so it has magnitude (the speed, 18 cm/s) and direction (the negative sign, indicating that it is moving toward the negative side of the pool table, away from you). This is also seen in the displacement, which is negative.
\begin{ProblemSub}
	b) How long does it take the cue ball to slide from $ x = 17 $ cm to $ x = -127 $ cm?
\end{ProblemSub}
We can start with the same relationship between velocity, displacement, and elapsed time, and rearrange:
\[
v = \frac{\Delta x}{\Delta t} \implies \Delta t = \frac{\Delta x}{v} = \frac{x_{f}-x_{i}}{v} = \frac{-127\text{ cm}-17\text{ cm}}{-18\text{ cm/s}} = \frac{-144\text{ cm}}{-18\text{ cm}} = 8\text{ s.}
\]
\begin{ProblemSub}
	c) The cue rebounds from the far edge without losing speed. What is its position 10 s after it is at $ x = -127 $ cm?
\end{ProblemSub}
Note that $ v = +18\text{ cm/s} $ now. We once again take our expression for velocity and rearrange:
\[
\begin{split}
	v & = \frac{x_{f}-x_{i}}{\Delta t} \\
	x_{f}-x_{i} & = v\Delta t \\
	x_{f} & = x_{i} + v\Delta t = -127\text{ cm} + (18\text{ cm/s})(10\text{ s}) = 53\text{ cm.}
\end{split}
\]
Position is also a vector, so this positive number tells us that the ball is on the positive end of the table (on the near side of the midpoint, relative to you).
\end{document}