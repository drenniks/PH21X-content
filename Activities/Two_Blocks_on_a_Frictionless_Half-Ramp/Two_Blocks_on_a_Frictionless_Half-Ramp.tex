\Problem{Two Blocks on a Frictionless Half-Ramp}{\TBFricHRamp}{
Consider the situation depicted below. Friction is negligible for the blocks and surface, and we will assume the strings and pulley are ideal (massless, frictionless, etc.).
}
\ProblemFig{\TBFricHRampFig}{
\centering
\includegraphics[scale=1.2]{\FileDepth/Activities/Two_Blocks_on_a_Frictionless_Half-Ramp/Blocks_on_a_Half-Tilt_Frictionless}
}
\ProblemSub{\TBFricHRampA}{
(a) Draw a free-body diagram for each block.
}
\Solution{\TBFricHRampASol}{I will indicate the half-ramp with the subscript $R$ and the string with the subscript $S$.

\begin{figure}[h]
	\centering
	\begin{tikzpicture}
		\FBDaxes{-2.5,0}{30}{Aaxes}
		\FBDbox[0.6]{Aaxes}{30}{Abox}{A}
		\FBDvectorXY{Aboxblq}{0,-1.5}{FGA}
		\node[anchor=east] at (FGAtip) {$\vec{F}^{g}_{AE}$};
		\FBDvectorMA{Aboxrcent}{0.75}{30}{FTA}
		\node[anchor=south] at (FTAtip) {$\vec{F}^{T}_{AS}$};
		\FBDvectorMA{Aboxtcent}{1.3}{120}{FNA}
		\node[anchor=east] at (FNAtip) {$\vec{F}^{N}_{AR}$};
		\draw[thick] (-2.5,-0.7) arc (270:300:0.7);
		\node[anchor=north] at (-2.3,-0.7) {$\theta$};
		\begin{scope}[shift={(-2.5,0)}]
			\draw[thick,dashed] ({-2*cos(30)},-1) -- ({-2*cos(30)+1},-1);
			\draw[thick] ({-2*cos(30)+0.7},-1) arc (0:30:0.7);
			\node[anchor=south west] at ({-2*cos(30)+0.7},-1) {$\theta$};
		\end{scope}
		\FBDaxes{2.5,0}{0}{Baxes}
		\FBDbox[0.6]{Baxes}{0}{Bbox}{B}
		\FBDvectorXY{Bboxbcent}{0,-1.5}{FGB}
		\node[anchor=west] at (FGBtip) {$\vec{F}^{g}_{BE}$};
		\FBDvectorXY{Bboxtcent}{0,1.5}{FNB}
		\node[anchor=west] at (FNBtip) {$\vec{F}^{N}_{BR}$};
		\FBDvectorXY{Bboxlcent}{-0.75,0}{FTB}
		\node[anchor=south] at (FTBtip) {$\vec{F}^{T}_{BS}$};
	\end{tikzpicture}
\end{figure}
}
\TeachingTips{
This is one problem where it is imperative \underline{not} to use $G$ for ``ground'' to indicate the surface beneath the blocks. If you do, the force on block A by the ground will have a very unfortunate symbol: $\vec{F}_{AG}$. It may be rather uncomfortable to accidentally write something that looks like a slur in front of your students.
}
\ProblemSub{\TBFricHRampB}{
(b) Indicate the Newton’s third law pairs.
}
\Solution{\TBFricHRampBSol}{There are no third law pairs. $\vec{F}^{T}_{AS}$ and $\vec{F}^{T}_{BS}$ are equal in magnitude, but they are not opposite in direction, and they are not directly between boxes A and B, instead going through the string and pulley between them.

However, since these two forces are equal in magnitude (due to the ideality of the pulley and the string), I will simplify my notation by giving their magnitudes a single symbol: $F^{T} = F^{T}_{AS} = F^{T}_{BS}$.
}
\ProblemSub{\TBFricHRampC}{
(c) Find the normal force on each block.
}
\Solution{\TBFricHRampCSol}{In order to do this, I will write out Newton's second law in the $y$-direction for both blocks. Neither is accelerating perpendicular to the surface, so the net force in this direction is zero.
\begin{align*}
F^{net}_{A,y} & = F^{N}_{AR,y}+F^{g}_{AE,y} & F^{net}_{B,y} & = F^{N}_{BR,y}+F^{g}_{BE,y} \\
0 & = F^{N}_{AR}-m_{A}g\cos\theta & 0 & = F^{N}_{BR}-m_{B}g \\
F^{N}_{AR} & = m_{A}g\cos\theta & F^{N}_{BR} & = m_{B}g
\end{align*}
}
\ProblemSub{\TBFricHRampD}{
(d) Find the acceleration of the two blocks.
}
\Solution{\TBFricHRampDSol}{Both blocks accelerate in the $x$-direction, and since they are attached, they must have the same magnitude of acceleration. Also, since their coordinate systems agree on which direction along the surface is $+x$, the accelerations will also share the same sign. As such, I will use a single symbol for both accelerations: $a = a_{A,x} = a_{B,x}$.
\begin{align*}
	F^{net}_{A,x} & = F^{T}_{AS,x}+F^{g}_{AE,x} & F^{net}_{B,x} & = F^{T}_{BS,x} \\
	m_{A}a_{A,x} & = F^{T}-m_{A}g\sin\theta & m_{B}a_{B,x} & = -F^{T} \\
	m_{A}a & = F^{T}-m_{A}g\sin\theta & m_{B}a & = -F^{T}
\end{align*}
Adding these two equations together gives us
\begin{align*}
	(m_{A}+m_{B})a & = F^{T}-m_{A}g\sin\theta + (-F^{T}) = -m_{A}g\sin\theta \\
	a & = -\frac{m_{A}}{m_{A}+m_{B}}g\sin\theta
\end{align*}
The negative sign lets us know that these blocks will accelerate to the left.

In the special case where $m_{B} \ll m_{A}$, the fraction $\frac{m_{A}}{m_{A}+m_{B}}$ approaches 1, and we get $a = g\sin\theta$, which is the acceleration we expect from an object freely sliding down a ramp (which we expect to be the limiting case when there is no additional mass attached to A). Conversely, if $m_{B} \gg m_{A}$, then $\frac{m_{A}}{m_{A}+m_{B}} \approx \frac{m_{A}}{m_{B}} \approx 0$, and the blocks don't move. This we also expect, as a massive enough block B should not allow block A to move it, thus preventing any sliding.
}
\ProblemSub{\TBFricHRampE}{
(e) Find the tension in the string.
}
\Solution{\TBFricHRampESol}{We already know from part (d) that $F^{T} = -m_{B}a$, so we can substitute our answer for the acceleration into this to obtain
\[
F^{T} = \frac{m_{A}m_{B}}{m_{A}+m_{B}}g\sin\theta.
\]
If we wanted to do more algebra, we could also start with both equations from (d) and solve them for $F^{T}$ instead of for $a$:
\begin{align*}
	m_{A}a & = F^{T}-m_{A}g\sin\theta & m_{B}a & = -F^{T} \\
	m_{A}m_{B}a & = m_{B}F^{T}-m_{A}m_{B}g\sin\theta & m_{A}m_{B}a & = -m_{A}F^{T}
\end{align*}
Subtracting the right hand equation from the left hand equation gives us
\begin{align*}
	m_{A}m_{B}a - m_{A}m_{B}a & = m_{B}F^{T}-m_{A}m_{B}g\sin\theta - (-m_{A}F^{T}) \\
	0 & = (m_{A}+m_{B})F^{T} - m_{A}m_{B}g\sin\theta \\
	F^{T} & = \frac{m_{A}m_{B}}{m_{A}+m_{B}}g\sin\theta.
\end{align*}
Our answer is positive (for all sensible angles from 0$^{\circ}$ to 90$^{\circ}$), which should be the case for a magnitude of a vector. Sometimes, when we do force analysis and get a negative number, it just tells us that the direction we assumed for a force is backward from what it actually is, but in this case, getting a negative magnitude would tell us that something was wrong. After all, tension cannot push.

For $m_{A} \gg m_{B}$, we obtain $F^{T} \approx \frac{m_{A}m_{B}}{m_{A}}g\sin\theta = m_{B}g\sin\theta$. In this situation (as I mentioned in part (d)), block A is accelerating at $g\sin\theta$ down the ramp, so the force on block B needs to be just perfect to get it accelerating at $g\sin\theta$ to keep up.

Conversely, for $m_{B} \gg m_{A}$, we obtain $F^{T} \approx \frac{m_{A}m_{B}}{m_{B}}g\sin\theta = m_{A}g\sin\theta$. In this situation, both blocks are stationary (B is too massive to move), so the tension needs to be strong enough to counteract the $x$-component of the force of gravity on A and hold it in place.
}