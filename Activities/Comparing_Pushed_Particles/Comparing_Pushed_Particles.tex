\Problem{Comparing Pushed Particles}{\ComPushP}{
Particle A has less mass than particle B. Both are pushed forward across a frictionless surface by equal forces for 1 s. Both start from rest.
}
\ProblemSub{\ComPushPA}{
(a) Compare the amount of work done on each particle. That is, is the work done on A greater than, less than, or equal to the work done on B? Explain.
}
\Solution{\ComPushPASol}{

The amount of work done by a force depends on the sizes of the force and the displacement of the object under that force. The same force is applied to both particles, but since A has less mass, it will accelerate more and have greater displacement. Therefore, the work done on A is greater.
}
\ProblemSub{\ComPushPB}{
(b) Compare the impulses delivered to particles A and B. Explain.
}
\Solution{\ComPushPBSol}{

Because the same net force is applied over the same elapsed time, the impulses are the same.
}
\ProblemSub{\ComPushPC}{
(c) Compare the final speeds of particles A and B. Explain.
}
\Solution{\ComPushPCSol}{

Let the particles move along the $ x $-axis so we may discuss the components of their momenta. The same impulse $ J_{x} $ is delivered to both particles, and since they both start from rest, we know $ p_{ix} = 0 $. This gives
\[
J_{x} = \Delta p_{x} = p_{fx} - \cancel{p_{ix}} = p_{fx}.
\]
As such, both particles have the same final momentum:
\[
p_{fx} = m_{A}v_{fA} = m_{B}v_{fB}.
\]
However, since $ m_{A} < m_{B} $, we must conclude that $ v_{fA} > v_{fB} $.
}