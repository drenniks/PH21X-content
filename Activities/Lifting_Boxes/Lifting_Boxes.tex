\Problem{Lifting Boxes}{\LiftBox}{
Rudy picks up a 5 kg box and lifts it straight up, at constant speed, to a height of 1 m. Beth uses a rope to pull a 5 kg box up a 15$ ^{\circ} $ frictionless slope, at constant speed, until it has reached a height of 1 m. Which of the two does more work? Or do they do equal amounts of work? Explain.
}
\Solution{\LiftBoxSol}{

The amount of work done is the same for both. To raise an object higher, one must do work against gravity equal to $ mgh $ independent of the path taken to reach the height $ h $.

Using a work approach, we know that Rudy must be exerting a force of $ mg = (5 \text{ kg})(9.8\text{ m/s}^{2}) = 49 $ N up (opposing gravity) to lift the box at constant speed. The displacement is $ h = 1 $ m, and force and displacement are in the same direction, so the work done is $ W = mgh = 49 $ J. When Beth pulls the box up the $ \theta = 15^{\circ} $ slope, she must pull with $ mg\sin\theta $ to oppose the force of gravity parallel to the slope. We know that the height of the slope is $ h $, and if the length of the slope (the hypotenuse of the triangle it forms with the vertical axis and the ground) is $ L $, then we know that $ \sin\theta = \frac{h}{L} $, and thus the displacement of the box $ L = \frac{h}{\sin\theta} $. The force and displacement are in the same direction, so the work done by Beth is $ W = (mg\sin\theta)L = mgh = 49 $ J. Again, both do the same amount of work. Beth just does it over a longer distance with a lesser force.
}
\TeachingTips{
Make sure that students walk through work in each case. This is also a valuable opportunity to spend more time on relating the angle of the ramp to the angle of the forces in one's coordinate system. There is the exact, geometrical approach of setting up the similar triangles, there is the visualization approach of thinking about how the coordinate system tilts with the ramp, and there is the special cases approach of seeing if the chosen angle makes sense when the ramp is horizontal or vertical (for example, seeing if the component of gravity down the ramp goes to zero when it is horizontal and to $mg$ when it is vertical).
}