\Problem{Free Fall from FBD}{\Falling}{
	Using Newton's 2nd law, show that if air resistance is negligible the acceleration of a freely falling object equals the acceleration due to gravity. [Hint: see your free-body diagram from Activity 1(b).]
}
\Solution{\FallingSol}{Newton's 2nd law states that the net force (the sum of \textbf{all} forces on an object) is proportional to the acceleration of the object. In particular,
	\[
	\vec{F}_{net} = m\vec{a}.
	\]
	In the specific case of an object in free fall, the only (non-negligible) force on the object is the force of gravity, $ \vec{F}_{g} = m\vec{g} $. As such, $ \vec{F}_{net} = m\vec{g} $, and we can apply Newton's 2nd law to see
	\[
	\begin{split}
		\cancel{m}\vec{a} & = \cancel{m}\vec{g} \\
		\vec{a} & = \vec{g}.
	\end{split}
	\]
	As such, acceleration in free fall has no dependence upon the mass of the object. Though a more massive object weighs more (has a greater force of gravity upon it), it also has more inertia, and its velocity is therefore more resistant to change.
}